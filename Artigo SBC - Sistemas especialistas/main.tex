\documentclass[12pt]{article}

\usepackage{sbc-template} 
\usepackage{graphicx,url}
\usepackage{url}
\usepackage[brazil]{babel} 
\usepackage[utf8]{inputenc} 
\usepackage[T1]{fontenc}
\usepackage[normalem]{ulem}
\usepackage[hidelinks]{hyperref}

\usepackage[square,authoryear]{natbib}
%\usepackage{amssymb} 
%\usepackage{mathalfa} 
%\usepackage{algorithm} 
%\usepackage{algpseudocode} 
%\usepackage[table]{xcolor}
%\usepackage{array}
\usepackage{titlesec}
%\usepackage{mdframed}
%\usepackage{listings}

%\usepackage{amsmath} 
%\usepackage{booktabs}

\urlstyle{same}

%\newcolumntype{L}[1]{>%{\raggedright\let\newline\\\arraybackslash\hspace{0pt}}m{#1}}
%\newcolumntype{C}[1]{>{\centering\let\newline\\\arraybackslash\hspace{0pt}}m{#1}}
%\newcolumntype{R}[1]{>{\raggedleft\let\newline\\\arraybackslash\hspace{0pt}}m{#1}}

%\newcommand\Tstrut{\rule{0pt}{2.6ex}} 
%\newcommand\Bstrut{\rule[-0.9ex]{0pt}{0pt}} 
%\newcommand{\scell}[2][c]{\begin{tabular}[#1]{@{}c@{}}#2\end{tabular}}

\usepackage[nolist,nohyperlinks]{acronym}

\title{Sistemas Especialistas}

\author{Autores: Matheus Mello e Jorge Harbes\inst{1}}


\address{Centro Federal de Educação Tecnológica Celso Suckow da Fonseca - CEFET/RJ
\email{jharbes@hotmail.com}
\email{matheusmello142012@gmail.com}
}


\begin{document} 
	
	\maketitle
	
	\begin{resumo} 
		Este artigo tem o propósito de reunir informações sobre as experiências e conhecimentos absorvidos nas aulas de sistemas especialistas, lecionada pelo professor \href{https://turing.pro.br/kadupantoja/}{Carlos Eduardo Pantoja}.
	\end{resumo}
	
	\section{Introdução}
	\label{sec:introducao}
		
    Nesta disciplina aprendemos um pouco mais sobre agentes, sistemas multiagentes, agentes cognitivos baseados em BDI e programação baseada em estímulos. O conteúdo é vasto, e as possibilidades, exponenciais. Como, por exemplo, um novo modelo de comunicação entre drones, abrindo novas possibilidades para esta tecnologia que hoje já se tornou um mercado.
    
    Utilizamos a IDE \href{https://github.com/chon-group/chonIDE}{ChonIDE}, desenvolvida pelos nossos próprios colaboradores.
	
	\section{O que é um agente ?}
	\label{sec:agente}
	
	Um agente é uma entidade autônoma e computacional que pode perceber seu ambiente, tomar decisões, agir e interagir com outros agentes para alcançar um ou mais objetivos. Normalmente é programado para que, diante de determinada análise, tome decisões baseado em conhecimentos preestabelecidos ou adquiridos. É o protagonista de todo o enredo de sistema multiagente.
	
	\section{Sistema MultiAgente}
	\label{sec:sis_multiagente}

    Um sistema multiagente é composto por vários agentes que interagem entre si para atingir objetivos individuais ou coletivos. Estes agentes podem ser tanto cooperativos quanto competitivos, dependendo da natureza do sistema e dos objetivos a serem alcançados. Cada agente em um sistema multiagente possui a capacidade de operar de forma independente, mas a interação entre os agentes é crucial para o funcionamento eficaz do sistema como um todo.
	
	\section{Agentes cognitivos em BDI}
	\label{sec:agentes_cog}

    BDI é uma abreviação de "Belief-Desire-Intention". É um modelo de arquitetura de agentes que é baseado nos conceitos de crenças (beliefs), desejos (desires) e intenções (intentions) de um agente.
    \begin{itemize}
    \item Crenças (Beliefs): São as informações que o agente possui sobre o mundo ao seu redor.
    \item Desejos (Desires): São os objetivos que o agente deseja alcançar.
    \item Intenções (Intentions): São os planos de ação que o agente decide executar para alcançar seus desejos.
    \end{itemize}
    
\begin{thebibliography}{9}
\bibitem{wooldridge}
M. Wooldridge,
\textit{An Introduction to MultiAgent Systems},
John Wiley \& Sons, 2009.

\bibitem{bratman}
M. E. Bratman,
\textit{Intentions, Plans, and Practical Reason},
Harvard University Press, 1987.
\end{thebibliography}

\end{document}
